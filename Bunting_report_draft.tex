\documentclass[11pt]{amsart}
\usepackage{geometry}           % See geometry.pdf to learn the layout options. There are lots.
\geometry{letterpaper} % ... or a4paper or a5paper  or ... 
%\geometry{landscape}                % Activate for for rotated page geometry
%\usepackage[parfill]{parskip}    % Activate to begin paragraphs with an empty line rather than an
%indent
\usepackage{graphicx}
\usepackage{amssymb}
\usepackage{epstopdf}
\usepackage{amsmath}
\DeclareGraphicsRule{.tif}{png}{.png}{`convert #1 `dirname #1`/`basename #1 .tif`.png}






\title{Tidal Asteroseismology}
\author{Andrew Bunting}
%\date{}                                           % Activate to display a given date or no date
%
\begin{document}

\maketitle

\section{Introduction}

Stars are ubiquitous in almost all areas of astrophysics, and it is becoming increasingly apparent that planets are a common occurrence as well.  My project revolves around the interaction between a subset of these two very common objects, and will enable us to test our understanding of what is going on inside the stars, and also could help in the hunt for planets.

\subsection{Asteroseismology}

Variable stars have been observed for over $3000$ years \cite{Jetsu2015}, although only in the last few centuries has the number of observed variable stars really grown \cite{Hoffleit1997}.  That stars are not constant was a major change from the classical view of the celestial sphere, and adds many layers to the complexity of stars, but also introduces new ways for us to learn about them.  Asteroseismology is the study of stellar oscillations, and is a rapidly growing field, as the quality of observations of stellar surfaces continues to improve.

Asteroseismology, understandably, first came about in studying the oscillations of our very own star -- potentially the first such vibrations were observed by Plaskett in 1916 \cite{Plaskett1916}, although the  observed variation in solar rotation was thought to be due to some atmospheric effects, but this was shown not to be the case by Hart in 1954 \cite{Hart1954}.  Since then, thousands of solar oscillation modes have been observed, each one enabling a subtly different probe of the solar interior \cite{DiMauro2017}.  The most prominent mode has a period of around $5$ minutes, and decays rapidly (over a few periods) \cite{Ulrich1970},  and is thought to be driven by convective motions inside the sun, although this is still an area of active research.  Observations of oscillations on other stars soon followed \cite{Brown1991}, including the detection of individual modes of oscillation on $\eta$ Bootis in 1995, although this was also unclear until it was confirmed in 2003 \cite{Kjeldsen2003}.

The great benefit in observing these osillations is the fact that they are oscillations throughout the body star, not merely at the surface.  This allows the interior structure of the star to be assessed much more readily than observations which depend only on the surface properties of the star.  This has been applied in various ways, including using asteroseismology to determine various properties of stars, including precise estimates of their ages \cite{Cunha2007} \cite{Chaplin2013}.










\bibliographystyle{plain}
\bibliography{library}






\end{document}  
